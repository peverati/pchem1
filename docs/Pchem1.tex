% Options for packages loaded elsewhere
\PassOptionsToPackage{unicode}{hyperref}
\PassOptionsToPackage{hyphens}{url}
%
\documentclass[
]{book}
\usepackage{lmodern}
\usepackage{amssymb,amsmath}
\usepackage{ifxetex,ifluatex}
\ifnum 0\ifxetex 1\fi\ifluatex 1\fi=0 % if pdftex
  \usepackage[T1]{fontenc}
  \usepackage[utf8]{inputenc}
  \usepackage{textcomp} % provide euro and other symbols
\else % if luatex or xetex
  \usepackage{unicode-math}
  \defaultfontfeatures{Scale=MatchLowercase}
  \defaultfontfeatures[\rmfamily]{Ligatures=TeX,Scale=1}
\fi
% Use upquote if available, for straight quotes in verbatim environments
\IfFileExists{upquote.sty}{\usepackage{upquote}}{}
\IfFileExists{microtype.sty}{% use microtype if available
  \usepackage[]{microtype}
  \UseMicrotypeSet[protrusion]{basicmath} % disable protrusion for tt fonts
}{}
\makeatletter
\@ifundefined{KOMAClassName}{% if non-KOMA class
  \IfFileExists{parskip.sty}{%
    \usepackage{parskip}
  }{% else
    \setlength{\parindent}{0pt}
    \setlength{\parskip}{6pt plus 2pt minus 1pt}}
}{% if KOMA class
  \KOMAoptions{parskip=half}}
\makeatother
\usepackage{xcolor}
\IfFileExists{xurl.sty}{\usepackage{xurl}}{} % add URL line breaks if available
\IfFileExists{bookmark.sty}{\usepackage{bookmark}}{\usepackage{hyperref}}
\hypersetup{
  pdftitle={The Live Textbook of Physical Chemistry 1},
  pdfauthor={Dr.~Roberto Peverati},
  hidelinks,
  pdfcreator={LaTeX via pandoc}}
\urlstyle{same} % disable monospaced font for URLs
\usepackage{longtable,booktabs}
% Correct order of tables after \paragraph or \subparagraph
\usepackage{etoolbox}
\makeatletter
\patchcmd\longtable{\par}{\if@noskipsec\mbox{}\fi\par}{}{}
\makeatother
% Allow footnotes in longtable head/foot
\IfFileExists{footnotehyper.sty}{\usepackage{footnotehyper}}{\usepackage{footnote}}
\makesavenoteenv{longtable}
\usepackage{graphicx,grffile}
\makeatletter
\def\maxwidth{\ifdim\Gin@nat@width>\linewidth\linewidth\else\Gin@nat@width\fi}
\def\maxheight{\ifdim\Gin@nat@height>\textheight\textheight\else\Gin@nat@height\fi}
\makeatother
% Scale images if necessary, so that they will not overflow the page
% margins by default, and it is still possible to overwrite the defaults
% using explicit options in \includegraphics[width, height, ...]{}
\setkeys{Gin}{width=\maxwidth,height=\maxheight,keepaspectratio}
% Set default figure placement to htbp
\makeatletter
\def\fps@figure{htbp}
\makeatother
\setlength{\emergencystretch}{3em} % prevent overfull lines
\providecommand{\tightlist}{%
  \setlength{\itemsep}{0pt}\setlength{\parskip}{0pt}}
\setcounter{secnumdepth}{5}
\usepackage{booktabs}
\usepackage{amsthm}
\makeatletter
\def\thm@space@setup{%
  \thm@preskip=8pt plus 2pt minus 4pt
  \thm@postskip=\thm@preskip
}
\makeatother
\usepackage{cancel}
\usepackage[]{natbib}
\bibliographystyle{apalike}

\title{The Live Textbook of Physical Chemistry 1}
\author{\href{mailto:rpeverati@fit.edu}{Dr.~Roberto Peverati}}
\date{01 August 2020}

\usepackage{amsthm}
\newtheorem{theorem}{Theorem}[chapter]
\newtheorem{lemma}{Lemma}[chapter]
\newtheorem{corollary}{Corollary}[chapter]
\newtheorem{proposition}{Proposition}[chapter]
\newtheorem{conjecture}{Conjecture}[chapter]
\theoremstyle{definition}
\newtheorem{definition}{Definition}[chapter]
\theoremstyle{definition}
\newtheorem{example}{Example}[chapter]
\theoremstyle{definition}
\newtheorem{exercise}{Exercise}[chapter]
\theoremstyle{remark}
\newtheorem*{remark}{Remark}
\newtheorem*{solution}{Solution}
\begin{document}
\maketitle

{
\setcounter{tocdepth}{1}
\tableofcontents
}
\centering

\begin{figure}
\centering
\includegraphics{./img/OEP_Figures.000.jpeg}
\caption{System}
\end{figure}

\hypertarget{preface}{%
\chapter*{Preface}\label{preface}}
\addcontentsline{toc}{chapter}{Preface}

\begin{center}\includegraphics[width=0.8\linewidth]{./img/OEP_Figures.000} \end{center}

This textbook is the official textbook for the Physical Chemistry 1 Course (CHM 3001) at Florida Tech.

The instructor for this course and author of this textbook is Dr.~Roberto Peverati.

CONTACTS: \href{mailto:rpeverati@fit.edu}{\nolinkurl{rpeverati@fit.edu}}, Office: OPS 333, (321) 674-7735

Chemistry Program, Department of Biomedical and Chenical Engineering and Science
Florida Institute of Technology, Melbourne, FL.

\begin{quote}
This live open textbook is distributed under the \href{https://creativecommons.org/licenses/by-sa/4.0/}{CC-BY-SA 4.0 License} and it was funded by the Florida Tech Open Educational Resources Grant Program: A Collaboration of the Teaching Council, eEducation, and the Evans Library.
\end{quote}

\hypertarget{how-to-use-this-book}{%
\section*{How to use this book}\label{how-to-use-this-book}}
\addcontentsline{toc}{section}{How to use this book}

Please read this book carefully, since everything that will be in your exams is explained here.
Since this book is specifically tailored for the CHM 3001 course at Florida Tech, there are no superfluous parts. In other words, everything in it might be subject to question in the quizzes and the final exam.

\begin{quote}
However, when you see materials in this format (lighter grey, indented, and following a grey vertical bar), it indicates a reinforcement paragraph that supports the main concept. In other words, these are additional explanations that might help you understand the main concepts better.
\end{quote}

Navigating the book should be straightforward. On each page, there is a useful sidebar on the left that gives you an overview of all chapters, and a toolbar at the top with important tools. Arrows to shift between chapters might also be present, depending on your browser. If you are old-school and prefer a pdf, you can download a printout by clicking on the toolbar's corresponding icon. If you are \emph{really} old-school and prefer a printed book, the best solution is to download the pdf and print it yourself. It is a LaTeX book, and I can promise you it will look good on paper. However, I cannot provide physical copies to each student. In the toolbar, you will find a useful search box that is capable of searching the entire book. The most adventurous will find in the toolbar a link to the raw GitHub source code. Feel free to head on \href{https://github.com/peverati/PChem1}{over there} and fork the book.

Each chapter of this book represents one week of work in the classroom and at home. The sidebar on the left will reflect your syllabus, as well as the main structure of the class on Canvas. The book is a live document, which means it will be updated throughout the semester with new material. While you are not required to check it every day, you might want to review each week's chapter before the lecture on Friday.

\begin{quote}
If you spot a mistake or a typo, contact Dr.~Peverati via \href{mailto:rpeverati@fit.edu}{email} and you will receive a credit of up to three points towards your final score, once the typo has been verified and corrected.
\end{quote}

\hypertarget{SystemVariables}{%
\chapter{Systems and Variables}\label{SystemVariables}}

\hypertarget{thermodynamic-systems}{%
\section{Thermodynamic Systems}\label{thermodynamic-systems}}

A thermodynamic system---or just simply a system---is a portion of space with defined boundaries that separate it from its surroundings (see also the title picture of this book). The surroundings may include other thermodynamic systems or physical systems that are not thermodynamic systems. A boundary may be a real physical barrier or a purely notional one. Typical examples of systems are reported in Figure \ref{fig:Fig1c1} below.

\begin{figure}

{\centering \includegraphics[width=0.8\linewidth]{./img/OEP_Figures.001} 

}

\caption{Examples of Thermodynamic Systems}\label{fig:Fig1c1}
\end{figure}

In the first case, a liquid is contained in a typical Erlenmeyer flask. The boundaries of the system are the glass walls of the beaker. The second system is represented by the gas contained in a balloon. The boundary is a physical barrier also in this case, being the plastic of the balloon. The third case is that of a thunder cloud. The boundary is not a well-defined physical barrier, but rather some condition of pressure and chemical composition at the interface between the cloud and the atmosphere. Finally, the fourth case is the case of an open flame. In this case, the boundary is again non-physical, and possibly even harder to define than for a cloud. For example, we can choose to define the flame based on some temperature threshold, or some color criterion, or even some chemical one. Despite the lack of physical boundaries, both the cloud and the flame---as portions of space containing matter---can be defined as a thermodynamic system.

A system can exchange exclusively mass, exclusively energy, or both mass and energy with its surroundings. Depending on the boundaries' ability to exchange these quantities, a system is defined as open, closed, or isolated. An open system exchanges both mass and energy. A closed system exchanges only energy, but not mass. Finally, an isolated system does not exchange mass nor energy.

\begin{longtable}[]{@{}lcc@{}}
\toprule
Type of System: & Mass & Energy (either heat or work)\tabularnewline
\midrule
\endhead
\textbf{Open} & Y & Y\tabularnewline
\textbf{Closed} & N & Y\tabularnewline
\textbf{Isolated} & N & N\tabularnewline
\bottomrule
\end{longtable}

When a system exchanges mass or energy with its surroundings, some of its parameters (variables) change. For example, if a system loses mass to the surroundings, the number of molecules (or moles) in the system will decrease. Similarly, if a system absorbs some energy, one or more of its variables (such as its temperature) increase. Mass and energy can flow into the system or out of the system. Let's consider mass exchange only. If some molecules of a substance leave the system, and then the same amount of molecules flow back into the system, the system will not be modified. We can count, for example, 100 molecules leaving a system and assign them the value of --100 in an outgoing process, and then observe the same 100 molecules going back into the system and assign them a value of +100. Regardless of the number of molecules present in the system in the first place, the overall balance will be --100 (from the outgoing process) +100 (from the ingoing process) = 0, which brings the system to its initial situation (mass has not changed). However, from a mathematical standpoint, we could have equally assigned the label +100 to the outgoing process and --100 to the ingoing one, and the overall total would have stayed the same: +100--100 = 0. Which of the two labels is best? For this case, it seems natural to define a mass going out of the system as negative (the system is losing it), and a mass going into the system as positive (the system is gaining it), but is it as straightforward for energy?

\begin{quote}
Here is another example. Let's consider a system that is composed of your body. When you exercise, you are losing mass in the form of water (sweat) and CO\textsubscript{2} (from respiration). This mass loss can be easily measured by stepping on a scale before and after exercise. The number you observe on the scale will go down. Hence you have lost weight. After exercise, you will reintegrate the lost mass by drinking and eating. If you have reintegrated the same amount you have lost, your weight will be the same as before the exercise (no weight loss). Nevertheless, which label do you attach to the amounts that you have lost and gained? Let's say that you are running a 5km race without drinking nor eating, and you measure your weight dropping 2 kg after the race. After the race, you drink 1.5 kg of water and eat a 500 g energy bar. Overall you did not lose any weight, and it would seem reasonable to label the 2 kg that you've lost as negative (--2) and the 1.5 kg of water that you drank and the 500 g bar that you ate as positive (+1.5 +0.5 = +2). But is it the only way? After all, you didn't gain nor lose any weight, so why not calling the 2 kg due to exercise +2 and the 2 that you've ingested as --2? It might seem silly, but mathematically it would not make any difference, the total would still be zero. Now, let's consider energy instead of mass. Let's say that in order to run the 5km race, you have spent 500 kcal, which then you reintegrate precisely by eating the energy bar. Which sign would you put in front of the kilocalories that you ``burned'' during the race? In principle, you've lost them, so if you want to be consistent, you should use a negative sign. But if you think about it, you've put quite an effort to ``lose'' those kilocalories, so it might not feel bad to assign them a positive sign instead. After all, it's perfectly OK to say, ``I've done a 500 kcal run today'', while it might sound quite awkward to say ``I've done a --500 kcal run today.'' Our previous exercise with mass demonstrates that it doesn't really matter which sign you put in front of the quantities. As long as you are consistent throughout the process, the signs will cancel out. If you've done a +500 kcal run, you've eaten a bar for --500 kcal, resulting in a total zero loss/gain. Alternatively, if you've done a --500 kcal run, you would have eaten a +500 kcal bar, for a total of again zero loss/gain.
\end{quote}

These simple examples demonstrate that the sign that we assign to quantities that flow through a boundary is arbitrary (i.e., we can define it any way we want, as long as we are always consistent with ourselves). There is no best way to assign those signs. If you ask two different people, you might obtain two different answers. But we are scientists, and we must make sure to be rigorous. For this reason, chemists have established a convention for the signs that we will follow throughout this course. If we are consistent in following the convention, we are guaranteed to never make any sign mistake.

\begin{definition}
\protect\hypertarget{def:chemistryconv}{}{\label{def:chemistryconv} }\emph{The chemistry convention of the sign is system-centric:}\footnote{Notice that physicists use a different sign convention when it comes to thermodynamics. To eliminate confusion, I will not describe the physics convention here, but if you are reading thermodynamics on a physics textbook, or if you are browsing the web and stumble on thermodynamics formula (e.g., on Wikipedia), please be advised that they might have a different sign than what is used in this course. Obviously, the science will not change, but you need to be ALWAYS consistent, so if you decide that you want to use the physics convention, make sure to ALWAYS use the physics convention. In this course, on the other hand, we will ALWAYS use the chemistry one, as introduced above.}
\end{definition}

\begin{itemize}
\tightlist
\item
  \emph{If something (energy or mass) goes \textbf{into} the system it has a \textbf{positive} sign (the system is gaining)}
\item
  \emph{If something (energy or mass) goes \textbf{out of} the system it has a \textbf{negative} sign (the system is losing)}
\end{itemize}

If you want a trick to remember the convention, use the weight loss/gain during the exercise example above. You are the system, if you lose weight, the kilograms will be negative (--2 kg), while if you gain weight, they will be positive (+2 kg). Similarly, if you eat an energy bar, you are the system, and you will have increased your energy by +500 kcal (positive). In contrast, if you burned energy during exercise, you are the system, and you will have lost energy, hence --500 kcal (negative). If the system is a balloon filled with gas, and the balloon is losing mass, you are the balloon, and you are losing weight; hence the mass will be negative. If the balloon is absorbing heat (likely increasing its temperature and increasing its volume), you are the system, and you are gaining heat; hence heat will be positive.

\hypertarget{thermodynamic-variables}{%
\section{Thermodynamic Variables}\label{thermodynamic-variables}}

The system is defined and studied using parameters that are called variables. These variables are quantities that we can measure, such as pressure and temperature. However, they don't be surprised if, on some occasions, you encounter some variable that is a little harder to measure directly, such as entropy. The variables depend only on the current state of the system, and therefore they define it. If I know the values of all the ``relevant variables'' of a system, I know the state of the system. The relationship between the variables is described by mathematical functions called state functions, while the ``relevant variables'' are called natural variables.

What are the ``relevant variables'' of a system? The answer to this question depends on the system, and it is not always straightforward. The simplest case is the case of an ideal gas, for which the natural variables are those that enter the ideal gas law and the corresponding equation:

\begin{equation}
  PV=nRT       
  \label{eq:idealgaslaworiginal}
\end{equation}

Therefore, the natural variables for an ideal gas are the pressure P, the volume V, the number of moles n, and the temperature T, with R being the ideal gas constant. Recalling from the general chemistry courses, R is a universal dimensional constant which has the values of R = 8.31 kJ/mol in SI units.\\
We will use the ideal gas equation and its variables as an example to discuss variables and functions in this chapter. We will analyze more complicated cases in the next chapters.
Variables can be classified according to numerous criteria, each with its advantages and disadvantages. A typical classification is:

\begin{itemize}
\tightlist
\item
  \textbf{Physical variables} (\(P\), \(V\), \(T\) in the ideal gas law): independent on the chemical composition of the system.
\item
  \textbf{Chemical variables} (\(n\) in the ideal gas law): dependent on the chemical composition of the system.
\end{itemize}

Another useful classification is:

\begin{itemize}
\tightlist
\item
  \textbf{Intensive variables} (\(P\), \(T\) in the ideal gas law): independent on the physical size (extension) of the system.
\item
  \textbf{Extensive variables} (\(V\), \(n\) in the ideal gas law): dependent on the physical size (extension) of the system.
\end{itemize}

When we deal with thermodynamic systems, it is more convenient to work with intensive variables. Luckily, it is relatively easy to convert extensive variables into intensive ones by just taking the ratio between two of them. For an ideal gas, by taking the ratio between V and n, we obtained the intensive variable called molar volume:

\begin{equation}
  V_m=\frac{V}{n}.   
  \label{eq:Vmdef}
\end{equation}

We can then recast eq. \eqref{eq:idealgaslaworiginal} as:

\begin{equation}
  PV_m=RT,
  \label{eq:idealgaslaw}
\end{equation}

which is the preferred equation that we will use for the remainder of this course.
The ideal gas equation connects the 3 variables pressure, molar volume, and temperature, and reduces the number of independent variables to just 2. In other words, once 2 of the 3 variables are known, the other one can be easily obtained using these simple relations:

\begin{equation}
  P(T,V_m)=\frac{RT}{V_m},
  \label{eq:PTVm}
\end{equation}

\begin{equation}
  V_m(T,P)=\frac{RT}{P},
  \label{eq:VmTP}
 \end{equation}

\begin{equation}
  T(P,V_m)=\frac{PV_m}{R}.
  \label{eq:TPVm}
\end{equation}

These equations define three state functions, each one expressed in terms of two independent natural variables. For example, eq. \eqref{eq:PTVm} defines the state function called ``pressure'', expressed as a function of temperature and molar volume. Similarly, eq. \eqref{eq:VmTP} defines the ``molar volume'' as a function of temperature and pressure, and eq. \eqref{eq:TPVm} defines the ``temperature'' as a function of pressure and molar volume. When we know the natural variables that define a state function, we can express the function using its total differential, for example for the pressure \(P(T, V_m)\):

\begin{equation}
  dP=\left( \frac{\partial P}{\partial T} \right)dT + \left( \frac{\partial P}{\partial V_m} \right)dV_m
  \label{eq:totaldifferentialP}
\end{equation}

Recalling Schwartz's theorem, the mixed partial second derivatives that can be obtained from eq. 1.2.7 are the same:

\begin{equation}
  \frac{\partial^2 P}{\partial T \partial V_m}=\frac{\partial}{\partial V_m}\frac{\partial P}{\partial T}=\frac{\partial}{\partial T}\frac{\partial P}{\partial V_M}=\frac{\partial^2 P}{\partial V_m \partial T}
  \label{eq:schwartzP}
\end{equation}

Which can be easily verified considering that:

\begin{equation}
  \frac{\partial}{\partial V_m} \frac{\partial P}{\partial T}  = \frac{\partial}{\partial V_m} \left(\frac{R}{V_m}\right) = -\frac{R}{V_m^2} 
  \label{eq:secondderPA}
\end{equation}

and

\begin{equation}
  \frac{\partial}{\partial T} \frac{\partial P}{\partial V_m}  = \frac{\partial}{\partial T} \left(\frac{-RT}{V_m^2}\right) = -\frac{R}{V_m^2} 
  \label{eq:secondderPB}
\end{equation}

While for the ideal gas law, all the variables are ``well-behaved'' and always satisfy Schwartz's theorem, we will encounter some variable for which Schwartz's theorem does not hold. Mathematically, if the Schwartz's theorem is violated (i.e., if the mixed second derivatives are not equal), then the corresponding function cannot be integrated, hence it is not a state function. The differential of a function that cannot be integrated cannot be defined exactly. Thus, these functions are called path functions; that is, they depend on the path rather than the state. The most typical example of path functions that we will encounter in the next chapters are heat (Q) and work (W). For these functions, we cannot define exact differentials \(dQ\) and \(dW\), and we must introduce a new notation to define their ``inexact differentials'' \(đ Q\) and \(đ W\).

\begin{quote}
We will return on exact and inexact differential when we discuss heat and work, but for this chapter, it is important to notice the difference between a state function and a path function. A typical example to understand the difference between state and path function is to consider the distance between two geographical locations. Let's, for example, consider the distance between New York City and Los Angeles. If we fly straight from one city to the other, there are roughly 4,000 km between them. This ``air distance'' depends exclusively on the geographical location of the two cities. It stays constant regardless of the method of transportation that I have accessibility to travel between them. Since the cities' positions depend uniquely on their latitudes and longitudes, the ``air distance'' is a state function, i.e., it is uniquely defined from a simple relationship between measurable variables. However, the ``air distance'' is not the distance that I will practically have to drive when I go from NYC to LA. Such ``travel distance'' depends on the method of transportation that I decide to take (airplane vs.~car vs.~train vs.~boat vs.~\ldots). It will depend on a plentiful amount of other factors such as the choice of road to be traveled (if going by car), the atmospheric conditions (if flying), and so on. A typical ``travel distance'' by car is, for example, about 4,500 km, which is about 12\% more than the ``air distance.'' Certainly, we could even design a very inefficient road trip that avoids all highways and will result in a ``travel distance'' of 8,000 km or even more (200\% of the ``air distance''). The ``travel distance'' is a clear example of a path function because it depends on the specific path that I decide to travel to go from NYC to LA. See Figure \ref{fig:Fig2c1}.
\end{quote}

\begin{figure}

{\centering \includegraphics[width=0.8\linewidth]{./img/OEP_Figures.002} 

}

\caption{State Functions vs. Path Functions}\label{fig:Fig2c1}
\end{figure}

\hypertarget{ZerothLaw}{%
\chapter{Zeroth Law of Thermodynamics}\label{ZerothLaw}}

\hypertarget{what-is-thermodynamics}{%
\section{What is Thermodynamics?}\label{what-is-thermodynamics}}

Thermodynamics is the branch of science that deals with heat and work, and their relation to energy. As the definition suggests, thermodynamics is concerned with two types of energy: heat and work. A definition of these forms of energy is as follow:

\begin{itemize}
\tightlist
\item
  Work is exchanged if external parameters are changed during the process.
\item
  Heat is exchanged if only internal parameters are changed during the process.
\end{itemize}

As we saw in Chapter \ref{SystemVariables} though, heat and work are not ``well-behaved'' quantities because they are path functions. While on the one hand it might be simple to measure experimentally the amount of heat and/or work, these measured numbers cannot be used to define the state of a system. Since heat and work are path functions, their values depend directly on the methods that are used to transfer them (their paths). Understanding and quantifying these energy transfers is the reason why thermodynamics was developed in the first place. The origin of thermodynamics dates back to the seventeenth century, a time when people began to use heat and work for technological applications. These early scientists needed a mathematical tool to understand how heat and work were related with each other, and how they were also related with the other variables that they were able to measure, such as temperature and volume.

Before we even discuss the definition of energy and how it relates to heat and work, it is important to introduce the important concept of temperature. Temperature is an intuitive concept that has a surprisingly complex definition. In fact with this course we will not even arrive at a rigorous definition of temperature, so you will not see it in this book. However, for all our purposes, it is not important to have a microscopic definition of temperature, as long as we have guarantee that this quantity can be measured in an unambiguous manner. In other words, we need a mathematical definition of temperature that will agree with the physical existence of thermometers.

\hypertarget{the-zeroth-law-of-thermodynamics}{%
\section{The Zeroth Law of Thermodynamics}\label{the-zeroth-law-of-thermodynamics}}

The mathematical definition that guaranteer that the thermal equilibrium is an equivalence relation is called the zeroth law of thermodynamics. The zeroth law of thermodynamics states that if two thermodynamic systems are each in thermal equilibrium with a third one, then they are in thermal equilibrium with each other. The law might appear trivial and possibly superfluous, but it is a fundamental requirement for the mathematical formulation of thermodynamics, so it needs to be stated. The zeroth law can be summarized by the following simple mathematical relation:

\begin{definition}
\protect\hypertarget{def:zerothlaw}{}{\label{def:zerothlaw} }\emph{Zeroth Law of Thermodynamics:} If \(T_A = T_B\), and \(T_B = T_C\), then \(T_A = T_C\).
\end{definition}

Notice that when we state the zeroth law, it appears intuitive. However, this is not necessarily the case. Let's, for example, consider a pot of boiling water at \(P=\) 1 atm. Its temperature, \(T_{H_2O}\), is about 373 K. Let's now immerse in this water a coin made of wood and another coin made of metal. After some sufficient time, the wood coin will be in thermal equilibrium with the water, and its temperature \(T_W = T_{H_2O}\). Similarly, the metal coin will also be in thermal equilibrium with the water, hence \(T_M = T_{H_2O}\). According to the zeroth law the temperature of the wood coin and that of the metal coin are exactly the same \(T_W = T_M\), even if they are not in direct contact with each other. Now here's the catch: since wood and metal transmit heat in different manners, if I take the coins out of the water and put them immediately in your hands, one of them will be very hot, but the other will burn you. So, if you had to guess the temperature of the two coins without a thermometer, and without knowing that they were immersed in boiling water, would you guess that they have the same temperature? Probably not.

\hypertarget{workint}{%
\section{Work in Theromdynamics}\label{workint}}

\hypertarget{free-expansion-at-constant-temperature}{%
\subsection{Free expansion at constant temperature}\label{free-expansion-at-constant-temperature}}

\begin{figure}

{\centering \includegraphics[width=0.5\linewidth]{./img/OEP_Figures.003} 

}

\caption{Free Expansion at Constant Temperature of an Ideal Gas}\label{fig:Fig1c3}
\end{figure}

Let's consider the situation in Figure \ref{fig:Fig1c3}, where a special beaker with a piston that is free to move is filled with an ideal gas. The beaker sits in a horizontal position on a desk, so the piston is not subject to any external forces\footnote{For this simple thought experiment, we will ignore any external force that is not significant. In other words, we will not consider the friction of the piston on the beaker walls or any other foreign influence.}. The piston is initially compressed to a position that is not in equilibrium \((i)\). After the process, the piston reaches a final equilibrium position \((f)\). We assume that the only force acting on the piston is the pressure of the ideal gas, \(P\). How do we calculate the work (\(W\)) performed by the system?

From basic physics, we recall that the infinitesimal amount of work associated with an object moving in space is given by the force acting on the object (\(F\)) multiplied by the infinitesimal amount it gets displaced (\(d h\)):

\begin{equation}
  đ W = - Fdh,
  \label{eq:Wphysics}
\end{equation}

where the negative sign comes from the chemistry sign convention, Definition \ref{def:chemistryconv}, since the work in Figure \ref{fig:Fig1c3} is \emph{performed} by the system (expansion). What kind of force is moving the piston? It is the force due to the pressure of the gas. Relying upon another definition from physics, pressure is the ratio between the force (\(F\)) and the area (\(A\)) that such force acts upon:

\begin{equation}
  P = F/A.
  \label{eq:Pphysics}
\end{equation}

Obtaining \(F\) from eq. \eqref{eq:Pphysics} and replacing it in eq. \eqref{eq:Wphysics}, we obtain:

\begin{equation}
  đ W = - P \underbrace{Adh}_{dV},
  \label{eq:Wphysics2}
\end{equation}

and considering that \(Adh\) (area times infinitesimal height) is the definition of an infinitesimal volume \(dV\), we obtain:

\begin{equation}
  đ W = - PdV,
  \label{eq:Wdef}
\end{equation}

If we want to calculate the amount of work performed by a system, \(W\), from eq. \eqref{eq:Wdef}, we need to recall that \(đ W\) is an inexact differential. As such, we cannot integrate it from initial to final as for the (exact) differential of a state function, because:

\begin{equation}
  \int_{i}^{f}đ W \neq W_f - W_i,
  \label{eq:Wdiff}
\end{equation}

but rather:

\begin{equation}
  \int_{\text{path}} đ W = W,
  \label{eq:Wdiff2}
\end{equation}

where the integration is performed along the \emph{path}. Using eq. \eqref{eq:Wdiff2}, we can integrate eq. \eqref{eq:Wdef} as:

\begin{equation}
  \int đ W = W = - \int_{i}^{f} PdV,
  \label{eq:Wint}
\end{equation}

where the integral on the left-hand side is taken along the path,\footnote{from here on we will replace the notation \(\int_{\text{path}}\) with the more convenient \(\int\) and we will keep in mind that the integral of an inexact differential must be taken along the path.} while the integral on the right-hand side can be taken between the initial and final states, since \(dV\) is a state function.
How do we solve the integral in eq. \eqref{eq:Wint}? The pressure in this process is not constant since it decreases throughout the process. Therefore \(P\) cannot be moved outside the integral. However, if our gas is ideal, we can calculate the pressure using the ideal gas law \(P=\frac{nRT}{V}\), and solve the integral because \(n\), \(R\), and \(T\) are constant:

\begin{equation}
  W = - nRT \int_{i}^{f} \frac{dV}{V} = -nRT \ln \frac{V_f}{V_i},
  \label{eq:WintsolvedV}
\end{equation}

which, considering that \(P_iV_i=P_fV_f\), can be also written as:

\begin{equation}
  W = -nRT \ln \frac{P_i}{P_f}.
  \label{eq:WintsolvedP}
\end{equation}

\hypertarget{isothermal-expansion-against-a-constant-external-pressure}{%
\subsection{Isothermal expansion against a constant external pressure}\label{isothermal-expansion-against-a-constant-external-pressure}}

\begin{figure}

{\centering \includegraphics[width=0.5\linewidth]{./img/OEP_Figures.004} 

}

\caption{Isothermal Expansion of an Ideal Gas Against a Constant External Pressure}\label{fig:Fig2c3}
\end{figure}

The process that we are analyzing here (Figure \ref{fig:Fig2c3}) is apparently similar to the case we have seen in the previous section, with the noticeably difference that the beaker now sits vertically on the workbench. For this case, when the piston moves upwards, it is no longer moving freely, but it moves against the force due to the constant external pressure \(P_{\text{ext}}\).

The integral that describes the work performed by the system in this case involves a transformation at constant external pressure:

\begin{equation}
  \int đ W = W = - \int_{i}^{f} P_{\text{ext}}dV,
  \label{eq:Wint2}
\end{equation}

which can be easily simplified as:

\begin{equation}
  W = - \int_{i}^{f} P_{\text{ext}}dV = -P_{\text{ext}} \int_{i}^{f} dV = -P_{\text{ext}} (V_f-V_i),
  \label{eq:Wint3}
\end{equation}

resulting in the following simple formula to calculate \(W\):

\begin{equation}
  W = -P_{\text{ext}} \Delta V,
  \label{eq:WintF}
\end{equation}

Comparing eq. \eqref{eq:WintF} with eq. \eqref{eq:WintsolvedP} shows how different this case is from the previous one.

\hypertarget{heatint}{%
\section{Heat in Theromdynamics}\label{heatint}}

Heat (Q) is a property that gets transferred between substances. Similarly to work, the amount of heat that flows through a boundary can be measured easily, but its mathematical interpretation is complicated because \emph{heat is a path function}.
As you probably recall from general chemistry, the ability of a substance to absorb heat is given by a coefficient called the heat capacity, which is measured in SI in \(\frac{\text{J}}{\text{mol K}}\). However, since heat is a path function, these coefficients are not unique, and we have different ones depending on how the heat transfer happens.

\hypertarget{processes-at-constant-volume-isochoric}{%
\subsection{Processes at constant volume (isochoric)}\label{processes-at-constant-volume-isochoric}}

The heat capacity at constant volume measures the ability of a substance to absorb heat at constant volume. Recasting from general chemistry:

\begin{quote}
The molar heat capacity at constant volume is the amount of heat required to increase the temperature of 1 mol of a substance by 1 K at constant volume.
\end{quote}

This simple definition can be written in mathematical terms as:

\begin{equation}
  C_V = \frac{đ Q_V}{n dT} \Rightarrow đ Q_V = n C_V dT.
  \label{eq:Cvdef}
\end{equation}

Given a known value of \(C_V\), the amount of heat that gets transfered can be easily calculated by measuring the changes in temperature, after integration of eq. \eqref{eq:Cvdef}:

\begin{equation}
  đ Q_V = n C_V dT \rightarrow \int đ Q_V = n \int_{T_i}^{T_F}C_V dT \rightarrow Q_V = n C_V \int_{T_i}^{T_F}dT,
  \label{eq:Cvint1}
\end{equation}

which, assuming \(C_V\) independent of temperature, simply becomes:

\begin{equation}
  Q_V \cong n C_V \Delta T.
  \label{eq:Cvint}
\end{equation}

\hypertarget{heatconstp}{%
\subsection{Processes at constant pressure (isobaric)}\label{heatconstp}}

Similarly to the previous case, the heat capacity at constant pressure measures the ability of a substance to absorb heat at constant pressure. Recasting again from general chemistry:

\begin{quote}
The molar heat capacity at constant pressure is the amount of heat required to increase the temperature of 1 mol of a substance by 1 K at constant pressure.
\end{quote}

And once again, this mathematical treatment follows:

\begin{equation}
  C_p = \frac{đ Q_p}{n dT} \Rightarrow đ Q_p = n C_p dT \rightarrow \int đ Q_p = n \int_{T_i}^{T_F}C_p dT,
  \label{eq:Cpdef}
\end{equation}

which result in the simple formula:

\begin{equation}
  Q_p \cong n C_p \Delta T.
  \label{eq:Cpint}
\end{equation}

\hypertarget{FirstLaw}{%
\chapter{First Law of Thermodynamics}\label{FirstLaw}}

\hypertarget{energy-in-thermodynamics}{%
\section{Energy in Thermodynamics}\label{energy-in-thermodynamics}}

The definition of energy is:

\begin{definition}
\protect\hypertarget{def:energy}{}{\label{def:energy} }\emph{Energy:} Property of a system that can be wither transferred or converted.
\end{definition}

In the absence of chemical transformations, heat and work are the only two forms of energy that thermodynamics is concerned with. Keeping in mind Definition \ref{def:chemistryconv}, which gives the convention for the signs of heat and work, the definition of energy can be written as:

\begin{equation}
  U = Q + W,
  \label{eq:U}
\end{equation}

which, in differential form, reads:
\begin{equation}
  dU = đ Q + đ W,
  \label{eq:dU}
\end{equation}

which, using eq. \eqref{eq:Wdef} becomes:

\begin{equation}
  dU = đ Q - PdV,
  \label{eq:dUpdv}
\end{equation}

\hypertarget{isothermic-processes-dt0}{%
\subsection{\texorpdfstring{Isothermic processes (\(dT=0\))}{Isothermic processes (dT=0)}}\label{isothermic-processes-dt0}}

To study the behavior of the energy at constant temperature, James Prescott Joule created the apparatus depicted in Figure \ref{fig:FigJexp}.

\begin{figure}

{\centering \includegraphics[width=0.8\linewidth]{./img/OEP_Figures.006} 

}

\caption{The Joule Expansion Experiment.}\label{fig:FigJexp}
\end{figure}

The left side of the Joule apparatus's inner chamber is filled with an ideal gas, while a vacuum is created in the right chamber. Both chambers are immersed in a water bath, to guarantee isolation from the environment. When the communication channel between the chambers is open, the gas expands and equilibrates. The work associated with the transformation is:

\begin{equation}
  đ W=P_{\text{ext}}dV = 0,
  \label{eq:JexpW}
\end{equation}

since the chambers are not in communication with the environment, \(P_{\text{ext}}=0\). Thus, changes in energy are associated with the heat transfer of the process, which can be measured by monitoring the temperature of the gas at the beginning, \(T_i\), and at the end of the experiment \(T_f\). Joule noticed experimentally that if he used an ideal gas for this experiment, the temperature will not change \(T_i = T_f\). Since the temperature doesn't change there is no heat transfer, and therefore the energy stays constant:

\begin{equation}
  dU = đ Q = 0.
  \label{eq:JexpQU}
\end{equation}

\begin{quote}
Notice that Joule's conclusion is valid only for an ideal gas. If we expand a real gas we do notice a change in temperature associated with the expansion. A typical example of this behavior is when you use a pressurized spray bottle and release its content for an extended time in the air. The container will typically get colder. We will discuss this behavior in a following chapter when we will study real gases.
\end{quote}

From this simple experiment, we can conclude that the energy of an ideal gas depends only on its temperature.

\hypertarget{adiabatic-processes-ux111q0}{%
\subsection{\texorpdfstring{Adiabatic Processes (\(đQ=0\))}{Adiabatic Processes (đQ=0)}}\label{adiabatic-processes-ux111q0}}

An adiabatic process is defined as a process that happens without the exchange of heat. As such, \(đ Q=0\), and the work associated with an adiabatic process becomes a state function:

\begin{equation}
  dU=đ W=PdV,
  \label{eq:dUadiabatic}
\end{equation}

which can then be calculated using the formulas that we derived in section \ref{workint}.

\hypertarget{isochoric-processes-dv0}{%
\subsection{\texorpdfstring{Isochoric processes (\(dV=0\))}{Isochoric processes (dV=0)}}\label{isochoric-processes-dv0}}

An isocoric process is a process in which the volume does not change. Therefore, \(đ W=0\), and \(dU = đ Q_V\), which for 1 mol of substance and using eq. \eqref{eq:Cvdef}, becomes:

\begin{equation}
  dU = đ Q_V = C_V dT.
  \label{eq:dUqv}
\end{equation}

Since no work is performed at these conditions, the heat becomes a state function. Eq. \eqref{eq:dUqv} also gives a mathematical justification of the concept of heat capacity at constant volume. \(C_V\) can now be interpreted as the partial derivative (a coefficient) of a state function (the energy):

\begin{equation}
  C_V = \left( \frac{\partial U} {\partial T} \right)_V,
  \label{eq:cvstatefunc}
\end{equation}

where we have replaced the total derivative \(d\) with a partial one \(\partial\), and we have specified that the derivation happens at constant volume. Eq. \eqref{eq:cvstatefunc} equation brings a rigorous definition of heat capacity at constant volume for 1 mol of substance:

\begin{definition}
\protect\hypertarget{def:newdefcv}{}{\label{def:newdefcv} }\emph{The heat capacity of a substance, \(C_V\), represents its ability to absorb \textbf{energy} at constant \textbf{volume}.}
\end{definition}

\hypertarget{isobaric-processes-dp0}{%
\subsection{\texorpdfstring{Isobaric processes (\(dP=0\))}{Isobaric processes (dP=0)}}\label{isobaric-processes-dp0}}

In an isobaric process, the pressure does not change, hence \(dP=0\). Unfortunately, eq. \eqref{eq:dU} for this case does not simplify further, as happened in the two previous cases. However, in section \ref{heatconstp}, we have introduced the useful concept of heat capacity at constant \(P\). \(C_P\) was used in an adiabatic process in the same manner as \(C_V\) was used in the isochoric case. That is, as a coefficient to measure the amount of heat absorbed at constant pressure. Eq. \eqref{eq:cvstatefunc} gave a mathematical definition of \(C_V\) as the partial derivative of a state function (the energy). But if heat capacities are coefficients, and coefficients are partial derivatives of state functions, how do we explain \(C_V\)?

In order to do so, we can introduce a new state function, called the enthalpy (\(H\)), as:

\begin{equation}
  H = U + PV,
  \label{eq:enthalpydef}
\end{equation}

and its differential, calculated as:

\begin{equation}
  dH = dU + d(PV) = dU + PdV + \overbrace{VdP}^{0},
  \label{eq:enthalpydefdiff}
\end{equation}

which can be rearranged as:

\begin{equation}
  dU = dH -PdV,
  \label{eq:enthalpydefdiffu}
\end{equation}

Replacing eq. \eqref{eq:enthalpydefdiffu} into eq. \eqref{eq:dUpdv}:

\begin{equation}
  dH -PdV = đ Q_P - PdV,
  \label{eq:dh1}
\end{equation}

which simplifies to:

\begin{equation}
  dH = đ Q_P.
  \label{eq:dh2}
\end{equation}

Eq. \eqref{eq:dh2} establishes that the heat exchanged at constant pressure is equal to a new state function called the enthalpy, defined by eq. \eqref{eq:enthalpydef}. It also establishes a mathematical justification of the concept of heat capacity at constant pressure. Similarly to \(C_V\), \(C_P\) can now be interpreted as the partial derivative (a coefficient) of the new state function (the enthalpy):

\begin{equation}
  C_P = \left( \frac{\partial H} {\partial T} \right)_P,
  \label{eq:cpstatefunc}
\end{equation}

Eq. \eqref{eq:cpstatefunc} brings also a rigorous definition of heat capacity at constant pressure for 1 mol of substance:

\begin{definition}
\protect\hypertarget{def:newdefcp}{}{\label{def:newdefcp} }\emph{The heat capacity of a substance, \(C_P\), represents its ability to absorb \textbf{enthalpy} at constant \textbf{pressure}.}
\end{definition}

\hypertarget{the-first-law-of-thermodynamics}{%
\section{The First Law of Thermodynamics}\label{the-first-law-of-thermodynamics}}

We finally come to a working definition of the first law. If we take an isolated system---i.e., a system that does not exchange heat nor mass with its surroundings---its energy is conserved. If the energy is conserved, \(dU=0\). Therefore, for an isolated system:

\begin{equation}
  đ Q = -đ W,
  \label{eq:heateqwork}
\end{equation}

and heat and work can be easily calculated using any of the appropriate formulas introduced in either Section \ref{workint} or \ref{heatint}.

The first law is a conservation law. It is intuitive since it comes directly from Lavoisier's principle of ``nothing is lost, nothing is created, everything is transformed.'' Considering that the only system that is truly isolated is the universe, we can condense the first law in one simple sentence:

\begin{definition}
\protect\hypertarget{def:firstlaw}{}{\label{def:firstlaw} }\emph{First Law of Thermodynamics:} The energy of the universe is conserved.
\end{definition}

\hypertarget{reversible-and-irreversible-processes}{%
\section{Reversible and Irreversible processes}\label{reversible-and-irreversible-processes}}

\hypertarget{calculation-of-w_textmax-and-w_textmin}{%
\subsection{\texorpdfstring{Calculation of \(| W_{\text{max}} |\) and \(| W_{\text{min}} |\)}{Calculation of \textbar{} W\_\{\textbackslash text\{max\}\} \textbar{} and \textbar{} W\_\{\textbackslash text\{min\}\} \textbar{}}}\label{calculation-of-w_textmax-and-w_textmin}}

Let's go back to the calculation of the work in a process at constant temperature. We can use the formulas obtained in section \ref{workint} to understand a little bit better the meaning of a \emph{path function}. Let's consider the following PV diagram, obtained from an ideal gas at constant \(T=298\) K:

\begin{center}\includegraphics{pchem1_files/figure-latex/unnamed-chunk-2-1} \end{center}

where the isothermal expansion happens between \(P_i\) and \(P_f\). If the expansion happens in a one-step fast process, for example against a constant pressure \(P_f=P_{\text{ext}}\), the work is given by eq. \eqref{eq:WintF}. On the plot, the absolute value of the work\footnote{we use the absolute value to avoid confusions due to the fact that the expansion work is negative according to Definition \ref{def:chemistryconv}.} is represented by the red area:

\begin{center}\includegraphics{pchem1_files/figure-latex/unnamed-chunk-3-1} \end{center}

\begin{equation}
\left| W_{\text{1-step}} \right| = P_{\text{ext}} (V_f-V_i)
  \label{eq:Warea1}
\end{equation}

However, if the process happens in two steps, by pausing at position (1) until equilibrium is reached, then we should calculate the work by dividing the process into two. The first process is an expansion between \(P\) and \(P_1\), whose absolute value of the work, \(W_A\), is represented by the blue area:

\begin{center}\includegraphics{pchem1_files/figure-latex/unnamed-chunk-4-1} \end{center}

\begin{equation}
\left| W_A \right| = P_1 (V_1-V_i)
  \label{eq:Warea2}
\end{equation}

The second process is an expansion between \(P_1\) and \(P_{\text{ext}}\), whose absolute value of the work is represented by the green area:

\begin{center}\includegraphics{pchem1_files/figure-latex/unnamed-chunk-5-1} \end{center}

\begin{equation}
\left| W_B \right| = P_f (V_f-V_1)
  \label{eq:Warea3}
\end{equation}

The total absolute value of the work for the 2-step process is given by the sum of the two areas:

\begin{center}\includegraphics{pchem1_files/figure-latex/unnamed-chunk-6-1} \end{center}

\begin{equation}
  \left| W_{\text{2-step}} \right| = \left| W_A \right| + \left| W_B \right| = P_1 (V_1-V_i)+P_f (V_f-V_1).
  \label{eq:Warea4}
\end{equation}

As can be easily verified by comparing the shaded areas in the plots, \(\left| W_{\text{2-step}} \right| > \left| W_{\text{1-step}} \right|\).

We can easily extend this procedure to consider processes that happens in 3, 4, 5, \ldots, \(n\) steps. What is the limit of this procedure? In other words, what happens when \(n \rightarrow \infty\)? A simple answer is given by the plots in the next Figure, which clearly demonstrates that the maximum value of the area underneath the curve \(\left| W_{\text{max}}\right|\) is achieved in an \(\infty\)-step process, for which the work is calculated as:

\begin{equation}
  \left| W_{\infty \text{-step}} \right| = \left| W_{\text{max}} \right| = \sum_{n}^{\infty} P_n(V_n-V_{n-1}) = \int_{i}^{f} PdV.
  \label{eq:WintsolvedV2}
\end{equation}

\includegraphics[width=0.5\linewidth,height=1\textheight]{pchem1_files/figure-latex/figures-side-1} \includegraphics[width=0.5\linewidth,height=1\textheight]{pchem1_files/figure-latex/figures-side-2} \includegraphics[width=0.5\linewidth,height=1\textheight]{pchem1_files/figure-latex/figures-side-3} \includegraphics[width=0.5\linewidth,height=1\textheight]{pchem1_files/figure-latex/figures-side-4}

The integral on the right hand side of eq. \eqref{eq:WintsolvedV2} is similar to the integral that we already solved for eq. \eqref{eq:WintsolvedV}. Using the same trick, we can solve eq. \eqref{eq:WintsolvedV2} for an ideal gas as:

\begin{equation}
  \left| W_{\text{max}} \right| = nRT \int_{i}^{f} \frac{dV}{V} = nRT \ln \frac{V_f}{V_i}.
  \label{eq:WmaxV}
\end{equation}

This example shows clearly why work is a path function. If we perform a fast 1-step expansion the system will perform an amount of work that is much smaller than the amount of work it can perform if the expansion between the same points happens slowly in an \(\infty\)-step process.

The same considerations that we made up to this point for expansion processes hold specularly for compression ones. The only difference is that the work associated with compressions will have a positive sign since it must be performed onto the system. As such, the amount of work for a transformation that happens in a finite amount of steps will be an upper bound to the minimum amount of work required to compress the system.\footnote{In contrast to a lower bound for expansion processes.} \(\left| W_{\text{min}} \right|\) for compressions is calculated as the area underneath the PV curve, exactly as \(\left| W_{\text{min}} \right|\) for expansions in eq. \eqref{eq:WintsolvedV2}.

\hypertarget{cycles-and-reversibility}{%
\subsection{Cycles and reversibility}\label{cycles-and-reversibility}}

Let's now consider the cycle in Figure \ref{fig:FigRevCyc}. The process in this case starts from state 1 (system at \(P_1V_1\)), expands to state 2 (system at \(P_2V_2\)), and compresses back to state 1 (system back to \(P_1V_1\)).

\begin{figure}

{\centering \includegraphics[width=0.8\linewidth]{./img/OEP_Figures.005} 

}

\caption{Expansion/Compression Cycle of an Ideal Gas}\label{fig:FigRevCyc}
\end{figure}

Since the process starts and finishes at the same state, the value of the energy at the end of the process will be the same as its value at the beginning, regardless of the path:\footnote{recall that the energy is a state function, so its value depend exclusively from the conditions at the beginning and at the end. In a cycle we're going back to the same point, so the conditions at the beginning and at the end are equal by definition.}

\begin{equation}
  \oint dE=0,
  \label{eq:de0}
\end{equation}

where the symbol \(\oint\) indicates an integral along a cycle. However, the same is not valid for work. For instance, if we perform the expansion in one step, the work associated with it will be (using eq. \eqref{eq:WintF}):\footnote{notice that the work for the expansion is negative, as it should be.}

\begin{equation}
  W^{\text{expansion}}_{\text{1-step}}=-P_2(\underbrace{V_2-V_1}_{>0})<0,
  \label{eq:Wexp1}
\end{equation}

and if we also perform the compression in 1-step:\footnote{notice that the work for the compression is positive, as it should be.}

\begin{equation}
  W^{\text{compression}}_{\text{1-step}}=-P_1(\underbrace{V_2-V_1}_{<0})>0.
  \label{eq:Wcomp1}
\end{equation}

With a little bit of math, it is easy to prove that the total work for the entire cycle is:

\begin{equation}
\begin{aligned}
W^{\text{cycle}}_{\text{1-step}} {} & =  W^{\text{expansion}}_{\text{1-step}}+W^{\text{compression}}_{\text{1-step}} \\
 & = -P_2(V_2-V_1)-P_1(V_1-V_2) \\
 & = -P_2(V_2-V_1)+P_1(V_2-V_1) \\
 & = (\underbrace{V_2-V_1}_{>0})(\underbrace{P_1-P_2}_{>0}) > 0,
\end{aligned}
  \label{eq:Wtot1}
\end{equation}

or, in other words, net work is destroyed.

\begin{quote}
In practice, if we want to manually perform this cycle by pushing on the piston by hand, we will notice that it requires more energy to push down than the amount it gives back when we release it and it moves back up.
\end{quote}

In contrast, if both the expansion and the compression happen in a slow \(\infty\)-step manner, the work associated with them will be \(W_{\text{max}}\) and \(W_{\text{min}}\), respectively, which are calculated using eq. \eqref{eq:WmaxV}. The total work related with the cycle will be in this case:

\begin{equation}
\begin{aligned}
W^{\text{cycle}}_{\infty\text{-step}} {} & = W^{\text{expansion}}_{\text{max}}+W^{\text{compression}}_{\text{min}} \\
 & = -nRT \ln \frac{V_f}{V_i}-nRT \ln \frac{V_i}{V_f} \\
 & = -nRT \left( \underbrace{\ln \frac{V_f}{V_i} - \ln \frac{V_f}{V_i}}_{=0} \right) = 0,
\end{aligned}
  \label{eq:Wtot2}
\end{equation}

which means that, in this case, work is not destroyed nor created.

\begin{quote}
In practice, if we were able to perform this cycle manually by pushing on the piston down by hand, we will notice that it requires the same amount of energy to push down than the amount it gives back when it moves up.
\end{quote}

This process can happen both ways without losses, and is called \emph{reversible}:

\begin{definition}
\protect\hypertarget{def:reversible}{}{\label{def:reversible} }\emph{Reversible process:} a process whose direction can be returned to its original position by inducing infinitesimal changes to some property of the system via its surroundings.\footnote{Definition from: Sears, F.W. and Salinger, G.L. (1986), Thermodynamics, Kinetic Theory, and Statistical Thermodynamics, 3rd edition (Addison-Wesley.)}
\end{definition}

Reversible processes are ideal processes that are hard to realize in practice since they require transformations that happen in an infinite amount of steps (infinitely slowly).

\hypertarget{Thermochemistry}{%
\chapter{Thermochemistry}\label{Thermochemistry}}

\hypertarget{ThermodynamicCycles}{%
\chapter{Thermodynamic Cycles}\label{ThermodynamicCycles}}

\begin{center}\includegraphics{pchem1_files/figure-latex/unnamed-chunk-7-1} \end{center}

\hypertarget{SecondThirdLaws}{%
\chapter{Second and Third Laws of Thermodynamics}\label{SecondThirdLaws}}

\hypertarget{Potentials}{%
\chapter{Chemical Potentials}\label{Potentials}}

\hypertarget{ChemicalEquilibrium}{%
\chapter{Chemical Equilibrium}\label{ChemicalEquilibrium}}

\hypertarget{Gases}{%
\chapter{Introduction to Gases}\label{Gases}}

\hypertarget{RealGases}{%
\chapter{Real Gases}\label{RealGases}}

\hypertarget{PhaseEquilibrium}{%
\chapter{Phase Equilibrium}\label{PhaseEquilibrium}}

\hypertarget{MCPhaseDiagrams}{%
\chapter{Multi-Component Phase Diagrams}\label{MCPhaseDiagrams}}

\hypertarget{Solutions}{%
\chapter{Solutions}\label{Solutions}}

\hypertarget{KTG}{%
\chapter{Kinetic Theory of Gas}\label{KTG}}

\hypertarget{Kinetics}{%
\chapter{Chemical Kinetics}\label{Kinetics}}

  \bibliography{book.bib,packages.bib}

\end{document}
